
% From https://www.overleaf.com/learn/latex/Glossaries

\makeglossaries % Prepare for adding glossary entries


\newglossaryentry{latex}
{
        name=latex,
        description={Is a mark up language specially suited for
scientific documents}
}

\newglossaryentry{bibliography}
{
        name=bibliography,
        plural=bibliographies,
        description={A list of the books referred to in a scholarly work,
typically printed as an appendix}
}

\newglossaryentry{maths}
{
    name=mathematics,
    description={Mathematics is what mathematicians do}
}


% --------------------
% ----- Acronyms -----
% --------------------

\newacronym[\glslongpluralkey={Generative Adversarial Networks}]{gan}{GAN}{Generative Adversarial Network}
\newacronym{cnn}{CNN}{Convolutional Neural Network}
\newacronym{ann}{ANN}{Artificial Neural Network}
\newacronym{ml}{ML}{Machine Learning}
\newacronym{ai}{AI}{Artificial Intelligence}
\newacronym{sgd}{SGD}{Stochastic Gradient Descent}

\newacronym{ct}{CT}{Computed Tomography}
\newacronym{fbp}{FBP}{Filtered Back Projection}
\newacronym{piccs}{PICCS}{Prior Image Constrained Compressed Sensing}

\newacronym{ssim}{SSIM}{Structural Similarity Index Measure}
\newacronym{mse}{MSE}{Mean Squared Error}
\newacronym{mae}{MAE}{Mean Absolute Error}
\newacronym{psnr}{PSNR}{Peak Signal-to-Noise Ratio}
\newacronym[\glslongpluralkey={Regions of Interest}]{roi}{ROI}{Region of Interest}