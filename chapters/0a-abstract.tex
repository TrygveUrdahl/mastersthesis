\chapter*{Abstract}
X-ray computed tomography (CT) allows for non-destructive imaging of internal structures of materials. The process of creating CT images involves recording x-ray projections of a sample, and computationally reconstructing the projections into a 3D image of the sample. Capturing the projections takes time, and introduces a source of radiation. Reducing the capture time and/or radiation dose creates noisy and artefact prone reconstructed images. 

A \textit{generative adversarial network} (GAN) has been trained to map noisy CT images to high-quality CT images, effectively denoising them. The GAN improves the structural similarity index measure (SSIM) of the noisy reconstruction from $0.233$ to $0.789$, and the mean squared error from $704.4$ to $210.8$, when denoising a projection undersampled reconstruction containing $46$ uniformly sampled projections from a high-quality reconstruction, which contains $1500$ projections. The GAN has been tested for a range of undersampling levels. 

A log-cosh term has been introduced to the loss function used to train the GAN, yielding an improvement in the achieved SSIM from $0.788$ to $0.789$ for the aforementioned undersampled reconstruction, without introducing any discernable drawbacks. 

The GAN denoising has been compared to a \textit{prior image constrained compressed sensing} (PICCS) reconstruction of a dynamic CT dataset. The GAN denoising achieves comparable image quality to the PICCS reconstruction, with some sample details not distinguishable in the PICCS reconstruction being captured by the GAN denoised reconstruction. 

Interaxial banding artefacts are introduced when denoising 2D slices of a 3D sample along an axial plane. These artefacts are reduced by using a depth parameter when training the GAN, allowing the denoising to utilize 3D spatial information from adjacent slices. 