\chapter{X-Ray Computed Tomography}
\label{sec:ct}
While this thesis is primarily focused on noise reduction using \acrshort{gan}s, a brief overview of some of the basic principles behind \acrfull{ct} will be given in this chapter. The focus will be on explaining the main sources of noise in \acrshort{ct} imaging. An overview of common reconstruction methods, as well as some novel ones, will also be presented.  

\todo[inline]{More on content of chapter}

\section{Theoretical Foundation}
X-ray \acrshort{ct} imaging is based on interaction between X-rays and matter. This interaction will attenuate X-rays that propagate through a sample according to the Beer-Lambert law, which is given as \cite{doi:10.1063/1.4950807}
\begin{equation}
    \label{eq:beerlambert}
    I = I_0 \exp^{-\int_{l_0}^{l}\mu\left(x,y,E\right)dl},
\end{equation}
where $I_0$ and $I$ is the incident and the attenuated X-ray beam intensities at positions $l_0$ and $l$ respectively, and $\mu$ is the attenuation coefficient of the traversed matter. The integral in the exponent is the path of the beam through the sample. The attenuation coefficient is dependent on energy ($E$), and typical X-ray \acrshort{ct} systems span a range of wavelengths (i.e. energies). 
Because of this, \cref{eq:beerlambert} must be modified to also account for the polychromatic nature of the X-ray source. 

The total incident radiation can be determined by integrating over all photon energies, 
\begin{equation}
    \label{eq:incidentradiation}
    I_0 = \int_{E_{min}}^{E_{max}}N\left(V,I\right)S\left(E\right)D\left(E\right)dE,
\end{equation}
with $N\left(V,I\right)$ being a variable introduced to account for photon flux depending on X-ray source tube voltage $V$ and current $I$, $S\left(E\right)$ being the normalized X-ray source spectrum modulated by the absorption materials between the source and the detector (not including the sample), and $D\left(E\right)$ being the detector sensitivity modulated by protection materials on the detector. $E_{min}$ and $E_{max}$ bound the energy range of the radiation spectrum. 

By combining \cref{eq:beerlambert,eq:incidentradiation}, we get the modified Beer-Lambert law accounting for the polychromatic X-rays \cite{doi:10.1063/1.4950807}, 
\begin{equation}
    I = \int_{E_{min}}^{E_{max}}N\left(V,I\right)S\left(E\right)D\left(E\right)\exp^{-\int_{l_0}^{l}\mu\left(x,y,E\right)dl}dE.
\end{equation}
This can be solved for the attenuation coefficient projection, giving \cite{doi:10.1063/1.4950807}
\begin{equation}
    \label{eq:ctattenuationcoefficient}
    \int_{l_0}^{l} \mu_m\left(x,y\right)dl = -\ln\left[\frac{\int S\left(E\right) D\left(E\right)\exp^{-\int_{l_0}^{l}\mu\left(x,y,E\right)dl} dE }{ \int S\left(E\right) D\left(E\right) dE }\right].
\end{equation}
The attenuated radiation $I_0$ can be related to the attenuation coefficient projection $\mu_m$ of a path through the sample by use of this equation. When the attenuation coefficient projections are known, the sample itself can be reconstructed using a reconstruction algorithm if a sufficient number of projections are available. 

\subsection{Noise}
Assuming a sufficient number of projections of the attenuation coefficient are available, the primary sources of noise in \acrshort{ct} measurements are quantum noise and electronic noise \cite{boas2012ct}. 

The quantum noise, sometimes also known as shot noise or simply Poisson noise, is due to the statistical error of low photon counts. It can be modeled as a Poisson distribution \cite{Whiting2006},
\begin{equation}
    \label{eq:poissonnoise}
    P(X = x) = \frac{e^{-xm}m^x}{x!},
\end{equation}
with $m$ being the mean signal value, $x=0,1,...$ being an integer representing the measured signal value, and $X$ being a random variable denoting the number of photons generated by the X-ray source. Quantum noise can be reduced simply by increasing the incident X-ray beam intensity, however this is often not wanted as increasing the radiation dose has raised conserns about potential health risks \cite{doi:10.1056/NEJMra072149,PEARCE2012499}. 

Electronic noise is related to the electronics of the X-ray detector, and it is modeled as additive white Gaussian (i.e. normal) noise \cite{boas2012ct},
\begin{equation}
    \mathcal{N}\left(0,\sigma \right) = \frac{1}{2\pi\sigma}\exp^{-\frac{x^2}{2\sigma}},
\end{equation}
which corresponds to a normal distribution $\mathcal{N}\left(\mu,\sigma\right)$ with mean signal value $\mu=0$ and standard deviation $\sigma$. 

If an insufficient number of projections of the attenuation coefficient are available, it is known as a missing wedge problem. Missing wedge measurements are incomplete datasets with respect to standard requirements of established reconstruction algorithms \cite{10.1111/jmi.12313}. This leads to noise artefacts that appear as elongations of reconstructed details along the mean direction (i.e. the symmetry centre of the projections). Several different methods of reducing this artefacting have been tried, including different reconstruction algorithms \cite{10.1111/jmi.12313} and machine learning based approaches \cite{liu2020tomogan}. 

\todo[inline]{Missing wedge artefacts}

\section{Imaging Method}
\subsection{Setup}


\section{Reconstruction}
\todo[inline]{Article explaining reconstruction methods: \cite{jimaging4110128}}

% Speed vs noise reduction
\subsection{Direct Reconstruction}
\todo[inline]{Filtered Back Projection (FBP)}
\todo[inline]{FDK?}
\todo[inline]{PICCS?}
\subsection{Iterative Reconstruction}
\todo[inline]{Simultaneous Iterative Reconstruction Technique (SIRT)}

\subsection{Other Methods}
\todo[inline]{GANrec}