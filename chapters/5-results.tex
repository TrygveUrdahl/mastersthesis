\chapter{Results and Discussion}
\label{sec:results}

\section{Effect of Pre-processing Images}
\todo[inline]{Compare tomo\_00058 denoising with and without cropping. Without cropping similar results to early stages of denoising cropped (i.e. poor and not converged). }

\cref{fig:uncroppeddenoising,fig:croppeddenoising}.

\begin{figure}[htbp]
  \centering
  \includegraphics[width=.9\textwidth]{figures/uncroppeddenoising.pdf}
  \caption[Non-cropped image denoising]{Denoising of non-cropped dataset tomo\_00058. Images d), e), and f) are zoomed in \acrshort{roi}s of images a), b), and c) respectively. The \acrshort{roi} is marked in a). }
  \label{fig:uncroppeddenoising}
\end{figure}

\begin{figure}[htbp]
  \centering
  \includegraphics[width=.9\textwidth]{figures/croppeddenoising.pdf}
  \caption[Cropped image denoising]{Denoising of cropped dataset tomo\_00058. Images d), e), and f) are zoomed in \acrshort{roi}s of images a), b), and c) respectively. The \acrshort{roi} is marked in a). }
  \label{fig:croppeddenoising}
\end{figure}


\section{Hyperparameter and Loss Function Changes}
\todo[inline]{Show difference in tomo\_00058 denoising for different loss functions / weights (and hyperparameters?)}

\todo[inline]{Line plot of gt, ns, MSE denoising, log-cosh denoising. Multiple lines around the images?}

\todo[inline]{SSIM and MSE evolution of arbitrary slice. }

\section{Different Amounts of Noise}
\todo[inline]{Different levels of noise / projection subsamplings (8,16,32,48). }
\todo[inline]{Plot of the images with zoomed in ROIs. }
\todo[inline]{Plot showing improvement in SSIM for different levels of noise. }
\todo[inline]{Line plot for different levels of noise? }

\cref{tab:missingwedgessim}, and \cref{fig:tomo00058missingwedgecomparison,fig:tomo00058missingwedgecomparisondenoised,fig:differentnoiselineplot1,fig:differentnoiselineplot2}. 
\todo[inline]{Keep one of \cref{tab:missingwedgessim,tab:missingwedgessim2}, which one is better? }

\begin{figure}
  \begin{subfigure}[t]{\textwidth}
    \centering
    \includegraphics[width=.45\textwidth]{figures/gt32.png}
    \caption{High-quality. }
  \end{subfigure}

  \medskip

  \begin{subfigure}[t]{.45\textwidth}
    \centering
    \includegraphics[width=\linewidth]{figures/ns8.png}
    \caption{Subsampling factor 8. }
  \end{subfigure}
  \hfill
  \begin{subfigure}[t]{.45\textwidth}
    \centering
    \includegraphics[width=\linewidth]{figures/ns16.png}
    \caption{Subsampling factor 16. }
  \end{subfigure}

  \medskip

  \begin{subfigure}[t]{.45\textwidth}
    \centering
    \includegraphics[width=\linewidth]{figures/ns32.png}
    \caption{Subsampling factor 32. }
  \end{subfigure}
  \hfill
  \begin{subfigure}[t]{.45\textwidth}
    \centering
    \includegraphics[width=\linewidth]{figures/ns48.png}
    \caption{Subsampling factor 48. }
  \end{subfigure}
  \caption[Four different levels of missing wedge noise]{Comparison of different levels of missing wedge noise on the tomo\_00058 dataset. The four noisy images have been reconstructed with \acrshort{fbp} with different levels of subsampling of the projections, leading to noise. The number of projections corresponding to a given subsampling factor can be seen in \cref{tab:projectionsubsampling}. }
  \label{fig:tomo00058missingwedgecomparison}
\end{figure}


\begin{figure}
  \begin{subfigure}[t]{.45\textwidth}
    \centering
    \includegraphics[width=\linewidth]{figures/ns8it100000itd4mse035logcosh3.png}
    \caption{Subsampling factor 8. }
  \end{subfigure}
  \hfill
  \begin{subfigure}[t]{.45\textwidth}
    \centering
    \includegraphics[width=\linewidth]{figures/ns16it100000itd4mse035logcosh3.png}
    \caption{Subsampling factor 16. }
  \end{subfigure}

  \medskip

  \begin{subfigure}[t]{.45\textwidth}
    \centering
    \includegraphics[width=\linewidth]{figures/ns32it100000itd4mse035logcosh3.png}
    \caption{Subsampling factor 32. }
  \end{subfigure}
  \hfill
  \begin{subfigure}[t]{.45\textwidth}
    \centering
    \includegraphics[width=\linewidth]{figures/ns48it100000itd4mse035logcosh3.png}
    \caption{Subsampling factor 48. }
  \end{subfigure}
  \caption[Denoising of four different levels of missing wedge noise]{Comparison of denoising of different levels of missing wedge noise on the tomo\_00058 dataset, where the corresponding noisy images can be seen in \cref{fig:tomo00058missingwedgecomparison}. The denoising was done with TomoGAN using a loss function containing \acrshort{mse}, log-cosh, VGG, and adversarial loss components, a depth of 1, and the network was trained for $100 000$ iterations. }
  \label{fig:tomo00058missingwedgecomparisondenoised}
\end{figure}

\begin{table}[htbp]
  \centering
  \caption[SSIM for different levels of simulated missing wedge noise and corresponding values after denoising]{Overview of \acrshort{ssim} for different levels of simulated missing wedge noise on the tomo\_00058 dataset. The missing wedge noise was simulated by subsampling the number of projections by a factor as given in the subsampling factor column, which results in a number of projections as given in \cref{tab:projectionsubsampling}. All denoising was done with TomoGAN using a loss function containing \acrshort{mse}, log-cosh, VGG, and adversarial loss components, a depth of 1, and the network was trained for $100 000$ iterations. }
  \label{tab:missingwedgessim}
  \begin{tabular}{lllll}
  \hline
  Subsampling factor & \multicolumn{2}{c}{Noisy values} & \multicolumn{2}{c}{Denoised values} \\
  \hhline{=====}
  {} & \acrshort{ssim} & \acrshort{mse} & \acrshort{ssim} & \acrshort{mse} \\
  \hline
  $1$  & $1.0$ & $0.0$ & $-$ & $-$ \\
  $8$  & $0.492$ & $148.3$ & $0.842$ & $33.3$ \\
  $16$ & $0.335$ & $348.5$ & $0.816$ & $74.9$ \\
  $32$ & $0.233$ & $704.4$ & $0.789$ & $210.8$ \\
  $48$ & $0.193$ & $976.6$ & $0.657$ & $2362.6$ \\
  \hline
  \end{tabular}
\end{table}

\begin{table}[htbp]
  \centering
  \caption[SSIM for different levels of simulated missing wedge noise and corresponding values after denoising]{Overview of \acrshort{ssim} for different levels of simulated missing wedge noise on the tomo\_00058 dataset. The missing wedge noise was simulated by subsampling the number of projections by a factor as given in the subsampling factor column, which results in a number of projections as given in \cref{tab:projectionsubsampling}. All denoising was done with TomoGAN using a loss function containing \acrshort{mse}, log-cosh, VGG, and adversarial loss components, a depth of 1, and the network was trained for $100 000$ iterations. }
  \label{tab:missingwedgessim2}
  \begin{tabular}{lllll}
  \hline
  Subsampling factor & \multicolumn{2}{c}{\acrshort{ssim}} & \multicolumn{2}{c}{\acrshort{mse}}  \\
  \hhline{=====}
  {} & Noisy & Denoised & Noisy & Denoised \\
  \hline 
  $1$  & $1.0$ & $-$ & $0.0$ & $-$ \\
  $8$  & $0.492$ & $0.842$ & $148.3$ & $33.3$ \\
  $16$ & $0.335$ & $0.816$ & $348.5$ & $74.9$ \\
  $32$ & $0.233$ & $0.789$ & $704.4$ & $210.8$ \\
  $48$ & $0.193$ & $0.657$ & $976.6$ & $2362.6$ \\
  \hline
  \end{tabular}
\end{table}

\begin{figure}[htbp]
  \centering
  \includegraphics[width=.95\textwidth]{figures/differentnoiselineplot1.pdf}
  \caption[Line plot of denoising of different levels of noise]{The plots show pixel values for 150 pixels on a horizontal line, as shown by the red line on the high-quality image above, for denoising of four different levels of missing wedge noise. HQ corresponds to the high-quality image, NS to the missing wedge noise image, and DN to the denoised image. All denoising was done with TomoGAN using a loss function containing \acrshort{mse}, log-cosh, VGG, and adversarial loss components, a depth of 1, and the network was trained for $100 000$ iterations. }
  \label{fig:differentnoiselineplot1}
\end{figure}

\begin{figure}[htbp]
  \centering
  \includegraphics[width=.95\textwidth]{figures/differentnoiselineplot2.pdf}
  \caption[Line plot of denoising of different levels of noise]{The plots show pixel values for 350 pixels on a horizontal line, as shown by the red line on the high-quality image above, for denoising of four different levels of missing wedge noise. Figure elements are as in \cref{fig:differentnoiselineplot1}. }
  \label{fig:differentnoiselineplot2}
\end{figure}

\section{Loss Function Evolution}
\todo[inline]{Plot how loss functions evolve through training of tomo\_00058 with good hyperparameters. }

\section{TomoGAN Compared to PICCS}
\todo[]{Change section name?}
\todo[inline]{Axial, sagittal, coronal plots for different depth parameters. }
\todo[inline]{Maybe make 3D model plot of this dataset. }
\todo[inline]{Line plot comparing GT, FDK?, PICCS, denoised. }
\todo[inline]{Histogram. Looks like peaks roughly align, sharper peaks. Looks like it is performing segmentation? Note: ordinate (y-axis) cropped to 20k. }

\cref{fig:kimrobertcomparison,fig:sideplothq,fig:sideplotfdk,fig:sideplotpiccs,fig:sideplotdepth1,fig:sideplotdepth3,fig:sideplotdepth5,fig:sideplotdepth7,fig:kimrobertline,fig:kimroberthist}. 

\begin{figure}
    \begin{subfigure}[t]{.45\textwidth}
      \centering
      \includegraphics[width=\linewidth]{figures/kimrobertgt.png}
      \caption{High-quality. }
    \end{subfigure}
    \hfill
    \begin{subfigure}[t]{.45\textwidth}
      \centering
      \includegraphics[width=\linewidth]{figures/kimrobertFDK.png}
      \caption{FDK.}
    \end{subfigure}
  
    \medskip
  
    \begin{subfigure}[t]{.45\textwidth}
      \centering
      \includegraphics[width=\linewidth]{figures/kimrobertPICCS.png}
      \caption{PICCS. }
    \end{subfigure}
    \hfill
    \begin{subfigure}[t]{.45\textwidth}
      \centering
      \includegraphics[width=\linewidth]{figures/kimrobertdepth1dn.png}
      \caption{FDK denoised. }
    \end{subfigure}
    \caption[Comparison of different reconstructions]{Comparison of different reconstructions. }
    \label{fig:kimrobertcomparison}
\end{figure}
\todo[]{Remove \cref{fig:kimrobertcomparison}? May not be needed, as this is basically showed in other figures. }

\begin{figure}[htbp]
  \centering
  \includegraphics[width=.85\textwidth]{figures/kimroberthq-x475y620s250.pdf}
  \caption[High-quality]{View of the high-quality reconstruction. The red horizontal line in \textbf{(a)} corresponds to the sagittal view in \textbf{(b)}, and the red vertical line corresponds to the coronal view in \textbf{(c)}. Three \acrshort{roi}s have been marked in \textbf{(b)} and \textbf{(c)}, and can be seen in \textbf{(d)}-\textbf{(f)}. }
  \label{fig:sideplothq}
\end{figure}

\begin{figure}[htbp]
  \centering
  \includegraphics[width=.85\textwidth]{figures/kimrobertfdk-x475y620s250.pdf}
  \caption[FDK]{View of the FDK reconstruction. Figure elements are as in \cref{fig:sideplothq}. }
  \label{fig:sideplotfdk}
\end{figure}

\begin{figure}[htbp]
  \centering
  \includegraphics[width=.85\textwidth]{figures/kimrobertpiccs-x475y620s250.pdf}
  \caption[PICCS]{View of the PICCS reconstruction. Figure elements are as in \cref{fig:sideplothq}. }
  \label{fig:sideplotpiccs}
\end{figure}

\begin{figure}[htbp]
  \centering
  \includegraphics[width=.85\textwidth]{figures/kimrobertdepth1-x475y620s250.pdf}
  \caption[Depth=1]{View of the denoised FDK reconstruction with a depth of 1. Figure elements are as in \cref{fig:sideplothq}. }
  \label{fig:sideplotdepth1}
\end{figure}

\begin{figure}[htbp]
  \centering
  \includegraphics[width=.85\textwidth]{figures/kimrobertdepth3-x475y620s250.pdf}
  \caption[Depth=3]{View of the denoised FDK reconstruction with a depth of 3. Figure elements are as in \cref{fig:sideplothq}. }
  \label{fig:sideplotdepth3}
\end{figure}

\begin{figure}[htbp]
  \centering
  \includegraphics[width=.85\textwidth]{figures/kimrobertdepth5-x475y620s250.pdf}
  \caption[Depth=5]{View of the denoised FDK reconstruction with a depth of 5. Figure elements are as in \cref{fig:sideplothq}. }
  \label{fig:sideplotdepth5}
\end{figure}

\begin{figure}[htbp]
  \centering
  \includegraphics[width=.85\textwidth]{figures/kimrobertdepth7-x475y620s250.pdf}
  \caption[Depth=7]{View of the denoised FDK reconstruction with a depth of 7. Figure elements are as in \cref{fig:sideplothq}. }
  \label{fig:sideplotdepth7}
\end{figure}

\begin{figure}[htbp]
  \centering
  \includegraphics[width=.85\textwidth]{figures/kimrobertline.pdf}
  \caption[Line plot]{The plot shows pixel values for 620 pixels on a horizontal line, as shown by the red line on the high-quality image above. The denoised values are from denoising the FDK reconstruction with TomoGAN using a loss function containing \acrshort{mse}, log-cosh, VGG, and adversarial loss components, a depth of 1, and the network was trained for $100 000$ iterations. }
  \label{fig:kimrobertline}
\end{figure}

\begin{figure}[htbp]
  \centering
  \includegraphics[width=.85\textwidth]{figures/kimroberthist.pdf}
  \caption[Histogram]{Histograms of different reconstructions, including a denoising of the FDK reconstruction. The denoising was done with TomoGAN using a loss function containing \acrshort{mse}, log-cosh, VGG, and adversarial loss components, a depth of 1, and the network was trained for $100 000$ iterations. Note that the ordinate has been cropped to a max value of $20000$. }
  \label{fig:kimroberthist}
\end{figure}

\section{Attempted Shale Denoising}
\todo[inline]{Include this?}
\todo[inline]{Shows limitations of method: requires a high-quality similar dataset (i.e. some ground truth) to work properly. Any given trained network doesn't work for all other dataset. }

Plot types: 
\begin{itemize}
    \item \acrshort{ssim} and \acrshort{mse} changes during training.
    \item Loss function evolution.
    \item Line plot of gt, ns, and different loss functions?
    \item Histograms of gt, ns, denoised
    \item Zoomed in region of interest.
    \item Axial, sagittal, and coronal plots of (at least Kim Robert's dataset) different depth parameters.
    \item Compare denoising of different subsamplings (8, 16, 32, 48)
    \item Activation plot of network layers.
\end{itemize}